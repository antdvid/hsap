\documentclass[12pt]{article}
\usepackage{multicol}
\textwidth=7in
\textheight=9.5in
\topmargin=-1in
\headheight=0in
\headsep=.5in
\hoffset  -.85in

\pagestyle{empty}
\renewcommand{\thefootnote}{\fnsymbol{footnote}}
\begin{document}
\begin{center}
{\bf High School Apprenticeship Program (HSAP)  

MWF 10:00 - 12:00 AM,

Room: Mathtower 1-125  
}
\end{center}

\setlength{\unitlength}{1in}

\begin{picture}(6,.1) 
\put(0,0) {\line(1,0){6.25}}         
\end{picture} 

\renewcommand{\arraystretch}{2}
\vskip.25in
\noindent\textbf{Description:}
HSAP provides junior and senior-level high school students with an authentic science
and engineering research experience to work alongside university researchers sponsored 
by the Army Research Office.
\vskip.25in

\noindent\textbf{Research Project at Stony Brook:}
Numerical Modeling of Fabric Surface and Simulation of Parachute Inflation in Turbulent Flow
\vskip.25in

\noindent\textbf{Program Coordinator:} Xiaolin Li,  Mathtower 1-122, 
xiaolin.li@stonybrook.edu
\vskip.25in

\noindent\textbf{Research Advisor:} Zheng Gao,  Mathtower 1-125, 
zheng.gao@stonybrook.edu
\vskip.25in

\noindent\textbf{Program Goals:}
\begin{itemize}
\item Provide participants with experience in developing and presenting scientific research;
\item Provide participants with experience to develop an independent research program in preparation for research fellowships, graduate school, and careers in science and engineering research;
\item Benefit from the expertise of a scientist or engineer as a mentor for professional and academic development purposes;
\item Develop students' skills and background to prepare them for competitive entry to science and engineering undergraduate programs.
\end{itemize}

\noindent \textbf{Course Outline:} 

\begin{center} \begin{minipage}{5in}
\begin{flushleft}
Linux basics \dotfill 1 week\\
C programming \dotfill 1 week\\
Math tools \dotfill 1 week\\
FronTier++ \dotfill 3 weeks\\
Parachute Simulation \dotfill 3 weeks\\
Report and presentation \dotfill 1 week\\
\end{flushleft}
\end{minipage}
\end{center}

\noindent\textbf{Report:}
Each student is required to write a 5-page report by the end of the program. The
content could be from the lectures during this program or extra materials that are
related to the project.
\vskip.25in

\noindent\textbf{Presentation:}
Each student is required to give a presentation at the end of the program. The
powerpoint is about 15 slides. The specific date and time will be announced two
weeks before the presentation.
\vskip.25in

\newpage

\section*{Linux Basics (1 week)}
\textbf{description:}
Get familiar with the basic Linux commands; learn to use vim to edit a file;
know how to use SSH.
\begin{multicols}{2}
\begin{itemize}
\item{About Linux and directory}
	\subitem{Set up the environment, passwd, ssh}
	\subitem{Listing files and directories}
	\subitem{changing directory}
	\subitem{the directories . and ..}
	\subitem{Pathnames}
	\subitem{Home directories and pathnames}

\item {About Files}
	\subitem{copying files}
	\subitem{moving files}
	\subitem{removing files}
	\subitem{cat,less,head}
	\subitem{grep,wc}

\item {Redirection}
	\subitem{who,sort}
	\subitem{Redirection}
	\subitem{Redirecting the output}
	\subitem{Redirecting the input}
	\subitem{Pipes}
	
\item {vi}
	\subitem{command mode and edit mode}
	\subitem{insert and replace}
	\subitem{copy, cut and paste}
	\subitem{visual mode}
	\subitem{search and go to}
	\subitem{search and replace}

\item {Wildcards}
	\subitem{Wildcards,*/?}
	\subitem{Filename conventitions:no space, underscore, extension}
	\subitem{getting help: man, whatis, history}

\item {Permission}
	\subitem{File system security}
	\subitem{changing access rights}
	\subitem{sleep}
	\subitem{processes and jobs}
	\subitem{Listing suspended and background processes}
	\subitem{Killing a process}

\item {Useful commands}
	\subitem{tar and untar}
	\subitem{who,top,w}
	\subitem{Environment variables}
	\subitem{Practice: install a software from source: "cacalib"}
\end{itemize} 
\end{multicols}

\newpage
\section*{C Programming (1 week)}
\textbf{description:}
Introduction to basic C programming under Linux environment; write
some simple C codes; debug with GDB.
\begin{multicols}{2}
\begin{itemize}
\item{Hello World}
	\subitem{Write Hello World Program}
	\subitem{Compile and run}

\item{Get to know C}
	\subitem{Token in C}
	\subitem{Variable}
	\subitem{Arithmetic operation}
	\subitem{Function}
	\subitem{Definition and declaration}
	
\item{Logical Structure}
	\subitem{logical comparison}
	\subitem{while loop}
	\subitem{for loop}
	\subitem{if, else if and else}
	\subitem{switch-case}
	\subitem{practice: sum of 1-100 using for and while}
	
\item{Array and pointer}
	\subitem{string and array}
	\subitem{pointer and address}
	\subitem{practice: swap two number using function}
	\subitem{practice: reverse string}

\item{Memory management}
	\subitem{alloc, malloc, relloc, free}
	\subitem{two dimensional array}
	\subitem{practice: print 1-16 using matrix}
	
\item{Interactive program}
	\subitem{scanf}
	\subitem{argc and argv}
	\subitem{practice: print 1-x using matrix, x is defined by user}	

\item{Recall compiling}
 	\subitem{compile and link}
 	\subitem{makefile}
 	\subitem{random number generation}
 	\subitem{practice: guess random number}
\end{itemize}
\end{multicols}

\newpage
\section*{Math Tools (1 week)}
\textbf{description:}
Learn how to use Octave/Matlab as a research tool.
\begin{itemize}
\item{Octave basics}
	\subitem{Math functions: sin, cos, tan, asin, acos, atan, log, log10, exp, abs}
	\subitem{linspace}
	\subitem{clc,clear,close}
	\subitem{plot, xlim, ylim, hold on}
	\subitem{run scripts}
	\subitem{Challenge: plot tan(x) between $[-\pi, \pi]$}
	
\item{Vector and matrix}
	\subitem{row and columns vector}
	\subitem{operator: $+, -, \ast, /, \hat{} $}
	\subitem{element operation: $.\ast, ./, .\hat{}$}
	\subitem{Index: A(i,j), A(:,j), A(1:end-1),j}
	\subitem{Ranges: start : step: end; 1 : 3 : 10}
	\subitem{Special matrix: tril(A), triu(A), eye(n), ones, zeros, rand, diag}
	\subitem{operation: fliplr, flipud, rot90. reshape}
	\subitem{Challenge: search diag and create a $20\times 20$ tridiagonal matrix}
	
\item{Plot}
	\subitem{2d plot: plot sin(x), linetype, color}
	\subitem{3d plot: meshgrid, surf}
	\subitem{save image: print, file type, jpg, eps}
	\subitem{Challenge: google and create a sphere?}
	
\end{itemize}

\newpage
\section*{FronTier++ (3 weeks)}
\textbf{description:}
Start working on FronTier++ library and run some two dimensional and three
dimensional examples; Learn some basic background about fluid dynamics;
know how to use VisIt to do post-processing of data from numerical simulations.
\begin{itemize}
\item{Set up environment}
	\subitem{Set up environment variable}
	\subitem{Install FronTier}
	\subitem{Compile and Run a simple 2d example}
	\subitem{Interpret the output: hdf, gview, vtk, run-output}
	\subitem{Challenge: Run a 3d example}

\item{Introduction to visualizing tools}
	\subitem{geomview}
	\subitem{xgraph}
	\subitem{hdf2gif, animate}
	\subitem{VisIt}
	
\item{Level set}
	\subitem{Get to know level set method}
	\subitem{Give a two dimensional example}
	\subitem{Write a level function for circle on the paper}
	\subitem{Insert your custom function}
	\subitem{Challenge: Generate a 3d cone}
	
\item{Introduction to Fluid dynamics}
	\subitem{Fluid properties: density, viscosity, Mach number, compressible and incompressible}
	\subitem{Video: supersonic and sound barrier}
	\subitem{Video: Turbulence structure}
	\subitem{Relation to mathematics and computation, Navier-Stokes equation.}

	\subitem{Homework: search and give a description of different parachutes:
	C9,G11,T10,T11,intruder}
\end{itemize}

\newpage
\section*{Parachute Simulation (3 weeks)}
\textbf{description:}
Get some background about the dynamic motion of the fabric surface using
the spring-mesh model; do some reference searching about parachute; run some
simulations of different types of parachutes.
\begin{itemize}
\item{Introduction to computational model of parachute}
	\subitem{Spring mass model}
	\subitem{Canopy surface and string}
	\subitem{Gore and vent}
	\subitem{Fluid dynamics}
\item{Introduction to input file}
	\subitem{Parameters}
\item{Run a simulation of parachute}
\item{Prepare data and make a movie}
	\subitem{Visit}
\item{Discussion about research topic}
\end{itemize}

\section*{Report and presentation (1 weeks)}
\textbf{description:}
Prepare for the presentation and write the report.
\end{document}